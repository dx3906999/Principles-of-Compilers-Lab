\documentclass[a4paper]{article}
\usepackage{ctex,amsmath,amssymb,ntheorem,framed,graphicx,subfigure,tablefootnote}
\usepackage{fancyhdr,array,pgfplots,capt-of}
\usepackage{fontspec}
\usepackage{xeCJK}
\usepackage{pgfplots}
\usepackage{capt-of}
\setCJKfamilyfont{huawen}{华文新魏}
\newcommand{\huawen}{\CJKfamily{huawen}}
\setCJKfamilyfont{hei}{SimHei}
\newcommand{\hei}{\CJKfamily{hei}}
\newcommand{\xiaosan}{\fontsize{15pt}{18pt}\selectfont}
\usepackage{minted}
\usepackage{listings}
\usepackage{xcolor}
\usepackage{geometry}
\geometry{a4paper,left=25mm,right=20mm,top=25mm,bottom=25mm}
\usepackage[colorlinks,linkcolor=black]{hyperref}
\newcommand{\upcite}[1]{\textsuperscript{\textsuperscript{\cite{#1}}}}

\newcommand{\course}{操作系统研讨课}

% \pagestyle{fancy}
% \lhead{\today}
% \chead{\course}
% \rhead{张荣宸 2023K8009907040}
% \renewcommand{\headrulewidth}{1pt}
% \renewcommand{\footrulewidth}{1pt}
% \usemintedstyle{vs}
\setmonofont{0xProto Nerd Font}
\setminted{frame=lines,bgcolor=gray!5,fontsize=\small,breaklines,framesep=2mm,mathescape,xleftmargin=2mm,xrightmargin=2mm,linenos=true,numbersep=8pt,gobble=0}

\newcounter{problemname}
\newenvironment{homework}[1][\stepcounter{problemname}\arabic{problemname}]{\begin{framed}\par\noindent\textbf{思考题#1. }}{\end{framed}\par}
\newenvironment{solution}{\par\noindent\textbf{答. }}{\par}
\newenvironment{note}{\par\noindent\textbf{注记. }}{\par}

\def\equationautorefname{式}
\def\footnoteautorefname{脚注}
\def\itemautorefname{项}
\def\figureautorefname{图}
\def\tableautorefname{表}
\def\partautorefname{篇}
\def\appendixautorefname{附录}
\def\chapterautorefname{章}
\def\sectionautorefname{节}
\def\subsectionautorefname{小小节}
\def\subsubsectionautorefname{subsubsection}
\def\paragraphautorefname{段落}
\def\subparagraphautorefname{子段落}
\def\FancyVerbLineautorefname{行}
\def\theoremautorefname{定理}

\begin{document}
% \begin{center}
%     \Large{\textbf{第一次实验}}
% \end{center}
\vspace*{8em}

\begin{center}
    \vspace*{6em}
    {\huawen\fontsize{32pt}{40pt}\selectfont 中国科学院大学}\\ [4em]
    {\huawen\fontsize{32pt}{40pt}\selectfont《编译原理》实验报告}\\ [3em]
    {\huawen\fontsize{20}{24pt}\selectfont 基于LLVM 的中间代码生成}\\ [3em]
\end{center}

\vfill

\begin{center}

\renewcommand{\arraystretch}{1.8}
\newcommand{\fixedunderlinecenter}[1]{\underline{\makebox[8cm][c]{#1}}}
\begin{tabular}{>{\raggedleft}p{3cm} p{10cm}}
{\hei\xiaosan 姓\qquad 名}: & \fixedunderlinecenter{{\hei\xiaosan 张荣宸、毕冠华}} \\
{\hei\xiaosan 学\qquad 号}: & \fixedunderlinecenter{{\hei\xiaosan 2023K8009907040、2023K8009970005}} \\
{\hei\xiaosan 专\qquad 业}: & \fixedunderlinecenter{{\hei\xiaosan 网络空间安全}} \\
{\hei\xiaosan 完成日期}: & \fixedunderlinecenter{{\hei\xiaosan\today}} \\
\end{tabular}
\end{center}

\newpage

\tableofcontents

\newpage

\section{实验小组成员及环境}

本实验由张荣宸(学号:2023K8009907040)和毕冠华(学号:2023K8009970005)共同完成。实验环境用docker构建镜像,并通过ssh进行连接。
\verb|Dockerfile|与第一次实验相同。



\section{语法树各节点的中间代码生成实现}

\subsection{obc检测}

obc检测到当前访问的数组下标越界时,调用运行时库函数\verb|obc_check_error|报告错误,本质上是添加一段代码进行动态检测。该函数定义如下:

\begin{minted}{cpp}
void SafeCIRBuilder::obc_check(llvm::Value *index, int array_length,
                               int node_line, int node_pos, std::string name)
{
    // Insert code to check if index is in [0, length).
    // If not in range, call obc_check_error to report error.
    // Basic logic:
    //          ... (current insert point)
    //          cmp = index < 0 || index >= length
    //          br cmp, check_fail, check_success
    //      check_fail bb:
    //          call obc_check(node_line, node_pos, name)
    //          ret void
    //      check_success bb:
    //          ... (next insert point here)

    // TODO: Implement.

    auto *fail_bb = llvm::BasicBlock::Create(context, "obc.fail", current_function);
    auto *cont_bb = llvm::BasicBlock::Create(context, "obc.cont", current_function);

    llvm::Value *idx32 = builder.CreateIntCast(index, builder.getInt32Ty(), true);
    auto *lt0 = builder.CreateICmpSLT(idx32, builder.getInt32(0));
    auto *ge_len = builder.CreateICmpSGE(idx32, builder.getInt32(array_length));
    auto *oob = builder.CreateOr(lt0, ge_len);
    builder.CreateCondBr(oob, fail_bb, cont_bb);

    builder.SetInsertPoint(fail_bb);
    // Call obc_check_error to report error:
    llvm::Value *arg0_val = llvm::ConstantInt::get(llvm::Type::getInt32Ty(context), node_line);
    llvm::Value *arg1_val = llvm::ConstantInt::get(llvm::Type::getInt32Ty(context), node_pos);
    llvm::Value *arg2_val = llvm::ConstantDataArray::getString(context, name);
    llvm::Value *arg0_ptr = lookup_variable("arg0").val_ptr;
    llvm::Value *arg1_ptr = lookup_variable("arg1").val_ptr;
    llvm::Value *arg2_ptr = lookup_variable("arg2").val_ptr;
    llvm::Function *check_err = functions["obc_check_error"];
    builder.CreateStore(arg0_val, arg0_ptr);
    builder.CreateStore(arg1_val, arg1_ptr);
    builder.CreateStore(arg2_val, arg2_ptr);
    builder.CreateCall(check_err, {});

    builder.CreateRetVoid();

    builder.SetInsertPoint(cont_bb);

    return;
}
\end{minted}

\subsection{cond节点的生成}

cond节点主要是插入了一条语句计算条件表达式的值作为左值。具体实现如下:

\begin{minted}{cpp}
void SafeCIRBuilder::visit(cond_node &node)
{
    // TODO: handle condition expression.
    llvm::Value *lhs_val;
    llvm::Value *rhs_val;
    EXPECT_RVAL(node.lhs->accept(*this));
    get_result_as_value(&lhs_val);
    EXPECT_RVAL(node.rhs->accept(*this));
    get_result_as_value(&rhs_val);

    llvm::Value *cond_val;
    switch (node.op)
    {
    case RelOp::EQUAL:
        cond_val = builder.CreateICmpEQ(lhs_val, rhs_val);
        break;
    case RelOp::NON_EQUAL:
        cond_val = builder.CreateICmpNE(lhs_val, rhs_val);
        break;
    case RelOp::LESS:
        cond_val = builder.CreateICmpSLT(lhs_val, rhs_val);
        break;
    case RelOp::LESS_EQUAL:
        cond_val = builder.CreateICmpSLE(lhs_val, rhs_val);
        break;
    case RelOp::GREATER:
        cond_val = builder.CreateICmpSGT(lhs_val, rhs_val);
        break;
    case RelOp::GREATER_EQUAL:
        cond_val = builder.CreateICmpSGE(lhs_val, rhs_val);
        break;

    default:
        break;
    }

    set_value_result(cond_val);
}
\end{minted}

\subsection{binop节点的生成}

binop节点主要分为两种情况:

\begin{itemize}
    \item 如果都是子节点都返回右值,则在该节点计算右值后返回右值。
    \item 如果其中有一个子节点返回左值,则生成计算指令然后返回左值。
\end{itemize}

具体实现如下:

\begin{minted}{cpp}
void SafeCIRBuilder::visit(binop_expr_node &node)
{
    // TODO: handle binary operation.
    /*
        Visit lhs and rhs, there can be 2 cases:
        1. If lhs and rhs are both constant, we can calculate the result directly and provide a
          constant result to the parent node:
            for example: "1+2" -> set_int_result(1+2)
        2. If lhs and rhs are not both constant, build an instruction for the binary op:
            for example: "a+1" -> %res = add %a, 1; set_value_result(res)
            for example: "a+b" -> %res = add %a, %b; set_value_result(res)
    */
    int lhs_int, rhs_int;
    bool lhs_is_int = false, rhs_is_int = false;
    llvm::Value *lhs_val = nullptr;
    llvm::Value *rhs_val = nullptr;

    EXPECT_RVAL(node.lhs->accept(*this));
    lhs_is_int = get_int_result(lhs_int);
    if (!lhs_is_int)
    {
        get_result_as_value(&lhs_val);
    }

    EXPECT_RVAL(node.rhs->accept(*this));
    rhs_is_int = get_int_result(rhs_int);
    if (!rhs_is_int)
    {
        get_result_as_value(&rhs_val);
    }

    if (lhs_is_int && rhs_is_int)
    {
        int res = 0;
        switch (node.op)
        {
        case BinOp::PLUS:
            res = lhs_int + rhs_int;
            break;
        case BinOp::MINUS:
            res = lhs_int - rhs_int;
            break;
        case BinOp::MULTIPLY:
            res = lhs_int * rhs_int;
            break;
        case BinOp::DIVIDE:
            res = lhs_int / rhs_int;
            break;
        case BinOp::MODULO:
            res = lhs_int % rhs_int;
            break;
        }
        set_int_result(res);
        return;
    }

    if (lhs_is_int)
    {
        lhs_val = builder.getInt32(lhs_int);
    }
    if (rhs_is_int)
    {
        rhs_val = builder.getInt32(rhs_int);
    }

    llvm::Value *res_val = nullptr;
    switch (node.op)
    {
    case BinOp::PLUS:
        res_val = builder.CreateAdd(lhs_val, rhs_val);
        break;
    case BinOp::MINUS:
        res_val = builder.CreateSub(lhs_val, rhs_val);
        break;
    case BinOp::MULTIPLY:
        res_val = builder.CreateMul(lhs_val, rhs_val);
        break;
    case BinOp::DIVIDE:
        res_val = builder.CreateSDiv(lhs_val, rhs_val);
        break;
    case BinOp::MODULO:
        res_val = builder.CreateSRem(lhs_val, rhs_val);
        break;
    }
    set_value_result(res_val);
}
\end{minted}

\subsection{unaryop节点的生成}

unaryop节点主要是对其子节点生成的右值进行一元操作,分类与上面类似,具体实现如下:

\begin{minted}{cpp}
void SafeCIRBuilder::visit(unaryop_expr_node &node)
{
    // TODO: handle unary operation.
    EXPECT_RVAL(node.rhs->accept(*this));

    int rhs_int;
    if (get_int_result(rhs_int))
    {
        switch (node.op)
        {
        case UnaryOp::PLUS:
            set_int_result(rhs_int);
            return;
        case UnaryOp::MINUS:
            set_int_result(-rhs_int);
            return;
        }
    }

    llvm::Value *rhs_val = nullptr;
    get_result_as_value(&rhs_val);
    llvm::Value *res_val = nullptr;
    switch (node.op)
    {
    case UnaryOp::PLUS:
        res_val = rhs_val;
        break;
    case UnaryOp::MINUS:
        res_val = builder.CreateNeg(rhs_val);
        break;
    }
    set_value_result(res_val);
}
\end{minted}

\subsection{lval节点的生成}

lval节点主要是访问变量或数组元素,访问obc数组元素时需要进行obc检测。在其访问数组元素时还增加了报错:如果访问数组元素时没有提供下标,则报错。
具体实现如下:

\begin{minted}{cpp}
void SafeCIRBuilder::visit(lval_node &node)
{
    auto name = node.name;
    VarInfo var_info = lookup_variable(name);
    if (!var_info.is_valid())
    {
        std::cerr << node.line << ":" << node.pos << ": variable '" << name
                  << "' is not declared" << std::endl;
        error_flag = true;
        return;
    }

    /*
        Use IS_EXPECT_LVAL() to check if the parent node expects an lval or an rval. And
          the parent must use EXPECT_LVAL() or EXPECT_RVAL() properly.
        If lval is expected, we can directly return the lval of the variable:
            for example: in "%a = 1" and we are visiting "%a",
            we can return the llvm::Value* of %a.
        If rval is expected, we need to load the value of the variable and return the loaded value.
            for example: in "1 + *%a" and we are visiting "%a",
            we need to return the loaded value of %a, create "%tmp = load %a" and then set_value_result(%tmp).
    */

    if (!var_info.is_array)
    {
        // TODO: handle scalar lval
        if (IS_EXPECT_LVAL())
        {
            set_value_result(var_info.val_ptr);
        }
        else
        {
            auto *loaded = builder.CreateLoad(builder.getInt32Ty(), var_info.val_ptr);
            set_value_result(loaded);
        }
    }
    else
    {
        // TODO: handle array lval and call obc_check to insert obc check code for obc array.
        if (!node.array_index)
        {
            std::cerr << node.line << ":" << node.pos << ": array index missing for '" << name << "'" << std::endl;
            error_flag = true;
            return;
        }

        EXPECT_RVAL(node.array_index->accept(*this));
        llvm::Value *idx_val = nullptr;
        get_result_as_value(&idx_val);
        auto *idx32 = builder.CreateIntCast(idx_val, builder.getInt32Ty(), true);

        if (var_info.is_obc)
        {
            obc_check(idx32, var_info.array_length, node.line, node.pos, name);
        }

        auto *arr_type = llvm::ArrayType::get(builder.getInt32Ty(), var_info.array_length);
        auto *elem_ptr = builder.CreateInBoundsGEP(arr_type, var_info.val_ptr,
                                                   {builder.getInt32(0), idx32});

        if (IS_EXPECT_LVAL())
        {
            set_value_result(elem_ptr);
        }
        else
        {
            auto *loaded = builder.CreateLoad(builder.getInt32Ty(), elem_ptr);
            set_value_result(loaded);
        }
        return;
    }
}
\end{minted}

\subsection{vardef节点的生成}

vardef节点主要是变量或数组的定义与初始化。在这里,笔者根据测例添加几处报错代码:

\begin{itemize}
    \item 全局变量初始化时,若初始化表达式不是常量表达式,则报错。
    \item 数组定义时,若数组长度不是常量表达式,则报错。
    \item 数组定义时,若数组长度不为正数,则报错。
    \item 数组定义时,若初始化表达式个数多于数组长度,则报错。
    \item 数组初始化时,若初始化表达式不是常量表达式,则报错。
    \item 变量重复定义时报错。
\end{itemize}

\begin{minted}{cpp}
void SafeCIRBuilder::visit(var_def_node &node)
{
    std::string name = node.name;
    bool is_const = node.is_const;
    bool is_obc = node.is_obc;
    ptr<expr_node> array_length = node.array_length;
    ptr_vector<expr_node> initializers = node.initializers;

    if (cur_scope == FLAG::GLOBAL_SCOPE)
    { // global define
        llvm::GlobalVariable *global_variable;

        if (!array_length)
        { // not an array
            // TODO: create and declare global scalar
            llvm::Constant *init = llvm::ConstantInt::get(builder.getInt32Ty(), 0);
            if (!initializers.empty())
            {
                EXPECT_RVAL(initializers[0]->accept(*this));
                int init_int;
                if (!get_int_result(init_int))
                {
                    std::cerr << node.line << ":" << node.pos << ": global init must be const" << std::endl;
                    error_flag = true;
                    return;
                }
                init = llvm::ConstantInt::get(builder.getInt32Ty(), init_int);
            }
            global_variable = new llvm::GlobalVariable(*module, builder.getInt32Ty(), is_const,
                                                       llvm::GlobalValue::ExternalLinkage, init, name);
            if (!declare_variable(name, global_variable, is_const, false, false, 0))
            {
                std::cerr << node.line << ":" << node.pos << ": variable '" << name
                          << "' is declared more than one times" << std::endl;
                error_flag = true;
                return;
            }
        }
        else
        { // is an array
            // TODO: create and declare global array
            EXPECT_RVAL(array_length->accept(*this));
            int len_const;
            if (!get_int_result(len_const))
            {
                std::cerr << node.line << ":" << node.pos << ": array length must be const" << std::endl;
                error_flag = true;
                return;
            }
            if (len_const <= 0)
            {
                std::cerr << node.line << ":" << node.pos << ": size of array '" << name << "' is not positive" << std::endl;
                error_flag = true;
                return;
            }
            if (static_cast<int>(initializers.size()) > len_const)
            {
                std::cerr << node.line << ":" << node.pos << ": excess elements in the initializer of array '" << name << "'" << std::endl;
                error_flag = true;
                return;
            }
            auto *arr_type = llvm::ArrayType::get(builder.getInt32Ty(), len_const);
            std::vector<llvm::Constant *> elems;
            elems.reserve(len_const);
            for (int i = 0; i < len_const; ++i)
            {
                if (i < static_cast<int>(initializers.size()))
                {
                    EXPECT_RVAL(initializers[i]->accept(*this));
                    int v;
                    if (!get_int_result(v))
                    {
                        std::cerr << node.line << ":" << node.pos << ": array init must be const" << std::endl;
                        error_flag = true;
                        return;
                    }
                    elems.push_back(llvm::ConstantInt::get(builder.getInt32Ty(), v));
                }
                else
                {
                    elems.push_back(llvm::ConstantInt::get(builder.getInt32Ty(), 0));
                }
            }
            auto *init_arr = llvm::ConstantArray::get(arr_type, elems);
            global_variable = new llvm::GlobalVariable(*module, arr_type, is_const,
                                                       llvm::GlobalValue::ExternalLinkage, init_arr, name);
            if (!declare_variable(name, global_variable, is_const, true, is_obc, len_const))
            {
                std::cerr << node.line << ":" << node.pos << ": variable '" << name
                          << "' is declared more than one times" << std::endl;
                error_flag = true;
                return;
            }
        }
    }
    else
    { // local define
        llvm::Value *local_variable;

        if (!array_length)
        { // not an array
            // TODO: create and declare local scalar
            local_variable = builder.CreateAlloca(builder.getInt32Ty(), nullptr, name);
            if (!initializers.empty())
            {
                EXPECT_RVAL(initializers[0]->accept(*this));
                llvm::Value *init_v;
                get_result_as_value(&init_v);
                builder.CreateStore(init_v, local_variable);
            }
            if (!declare_variable(name, local_variable, is_const, false, false, 0))
            {
                std::cerr << node.line << ":" << node.pos << ": variable '" << name
                          << "' is declared more than one times" << std::endl;
                error_flag = true;
                return;
            }
        }
        else
        { // is an array
            // TODO: create and declare local array
            EXPECT_RVAL(array_length->accept(*this));
            int len_const;
            if (!get_int_result(len_const))
            {
                std::cerr << node.line << ":" << node.pos << ": array length must be const" << std::endl;
                error_flag = true;
                return;
            }
            if (len_const <= 0)
            {
                std::cerr << node.line << ":" << node.pos << ": size of array '" << name << "' is not positive" << std::endl;
                error_flag = true;
                return;
            }
            if (static_cast<int>(initializers.size()) > len_const)
            {
                std::cerr << node.line << ":" << node.pos << ": excess elements in the initializer of array '" << name << "'" << std::endl;
                error_flag = true;
                return;
            }
            auto *arr_ty = llvm::ArrayType::get(builder.getInt32Ty(), len_const);
            local_variable = builder.CreateAlloca(arr_ty, nullptr, name);
            auto *zero_init = llvm::ConstantAggregateZero::get(arr_ty);
            builder.CreateStore(zero_init, local_variable);
            for (int i = 0; i < static_cast<int>(initializers.size()) && i < len_const; ++i)
            {
                EXPECT_RVAL(initializers[i]->accept(*this));
                llvm::Value *init_v;
                get_result_as_value(&init_v);
                auto *elem_ptr = builder.CreateInBoundsGEP(arr_ty, local_variable,
                                                           {builder.getInt32(0), builder.getInt32(i)});
                builder.CreateStore(init_v, elem_ptr);
            }
            if (!declare_variable(name, local_variable, is_const, true, is_obc, len_const))
            {
                std::cerr << node.line << ":" << node.pos << ": variable '" << name
                          << "' is declared more than one times" << std::endl;
                error_flag = true;
                return;
            }
        }
    }
}
\end{minted}

\subsection{assign节点的生成}

assign节点主要是对左值进行赋值操作,如果对应变量为常量则报错。赋值语句target期待左值,因为要将对应的值写入地址。具体实现如下:

\begin{minted}{cpp}
void SafeCIRBuilder::visit(assign_stmt_node &node)
{
    // TODO: get target's rval and store at value's lval.
    VarInfo target_info;
    target_info = lookup_variable(node.target->name);
    if (target_info.is_const)
    {
        std::cerr << node.line << ":" << node.pos << ": assignment of read-only variable '"
                  << node.target->name << "'" << std::endl;
        error_flag = true;
        return;
    }
    llvm::Value *value_val;
    llvm::Value *target_ptr;
    EXPECT_RVAL(node.value->accept(*this));
    get_result_as_value(&value_val);
    EXPECT_LVAL(node.target->accept(*this));
    get_value_result(&target_ptr);
    builder.CreateStore(value_val, target_ptr);
}
\end{minted}

\subsection{if节点的生成}

if节点主要有四个基本块:条件判断块、then块、else块和结束块。条件判断块中若没有终止指令,需要添加跳转到结束块。具体实现如下:

\begin{minted}{cpp}
void SafeCIRBuilder::visit(if_stmt_node &node)
{
    // TODO: implement if-else statement.
    auto cond_basic_block = llvm::BasicBlock::Create(context, "if.cond", current_function);
    auto if_basic_block = llvm::BasicBlock::Create(context, "if.body", current_function);
    auto else_basic_block = node.else_body ? llvm::BasicBlock::Create(context, "else.body", current_function) : nullptr;
    auto end_basic_block = llvm::BasicBlock::Create(context, "if.end", current_function);
    builder.CreateBr(cond_basic_block);

    builder.SetInsertPoint(cond_basic_block);
    EXPECT_RVAL(node.cond->accept(*this));
    llvm::Value *cond_val;
    get_result_as_value(&cond_val);
    if (!cond_val->getType()->isIntegerTy(1))
    {
        cond_val = builder.CreateICmpNE(cond_val, builder.getInt32(0));
    }
    builder.CreateCondBr(cond_val, if_basic_block, else_basic_block ? else_basic_block : end_basic_block);

    builder.SetInsertPoint(if_basic_block);
    if (node.if_body)
    {
        node.if_body->accept(*this);
    }
    auto *cur_bb = builder.GetInsertBlock();
    if (cur_bb && !cur_bb->getTerminator())
    {
        builder.CreateBr(end_basic_block);
    }

    if (else_basic_block)
    {
        builder.SetInsertPoint(else_basic_block);
        if (node.else_body)
        {
            node.else_body->accept(*this);
        }
        cur_bb = builder.GetInsertBlock();
        if (cur_bb && !cur_bb->getTerminator())
        {
            builder.CreateBr(end_basic_block);
        }
    }

    builder.SetInsertPoint(end_basic_block);
}
\end{minted}

\subsection{while节点的生成}

while节点主要有三个基本块:条件判断块、循环体块和结束块。条件判断块中若没有终止指令,需要添加跳转到结束块。具体实现如下:

\begin{minted}{cpp}
void SafeCIRBuilder::visit(while_stmt_node &node)
{
    // TODO: implement while statement.
    auto cond_basic_block = llvm::BasicBlock::Create(context, "while.cond", current_function);
    auto body_basic_block = llvm::BasicBlock::Create(context, "while.body", current_function);
    auto end_basic_block = llvm::BasicBlock::Create(context, "while.end", current_function);

    builder.CreateBr(cond_basic_block);

    builder.SetInsertPoint(cond_basic_block);
    EXPECT_RVAL(node.cond->accept(*this));
    llvm::Value *cond_val;
    get_result_as_value(&cond_val);
    if (!cond_val->getType()->isIntegerTy(1))
    {
        cond_val = builder.CreateICmpNE(cond_val, builder.getInt32(0));
    }
    builder.CreateCondBr(cond_val, body_basic_block, end_basic_block);

    builder.SetInsertPoint(body_basic_block);
    if (node.body)
    {
        node.body->accept(*this);
    }

    auto *cur_bb = builder.GetInsertBlock();
    if (cur_bb && !cur_bb->getTerminator())
    {
        builder.CreateBr(cond_basic_block);
    }

    builder.SetInsertPoint(end_basic_block);
}
\end{minted}

\subsection{查找变量的实现}

在查找变量时,需要从当前作用域开始,向上查找最近的同名变量。具体实现如下:

\begin{minted}{cpp}
SafeCIRBuilder::VarInfo SafeCIRBuilder::lookup_variable(std::string name)
{
    // TODO: find the nearest decalred variable `name`
    for (auto it = scoped_variables.rbegin(); it != scoped_variables.rend(); ++it)
    {
        auto found = it->variable_map.find(name);
        if (found != it->variable_map.end())
        {
            return found->second;
        }
    }
    return VarInfo();
}
\end{minted}

\section{彩蛋行为分析}

将\verb|check.py| 中加入 \verb|trick.c| 文件并运行,得到如\autoref{fig_1}

\begin{figure}[!htbp]
    \centering
    \includegraphics[width=0.7\linewidth]{./pic/屏幕截图 2026-01-04 150122.png}
    \caption{trick报错}
    \label{fig_1}
\end{figure}

笔者直接使用 \verb|irbuilder| 进行编译,生成的中间代码如下:

\begin{minted}{llvm}
; ModuleID = 'trick.c'
source_filename = "trick.c"

@input_var = global i32 0
@output_var = global i32 0
@arg0 = global i32 0
@arg1 = global i32 0
@arg2 = global [17 x i8] c"SafeC IR Builder\00"

declare void @input_impl(ptr)

declare void @output_impl(ptr)

declare void @obc_check_error_impl(ptr, ptr, ptr)

define void @input() {
entry:
  call void @input_impl(ptr @input_var)
  ret void
}

define void @output() {
entry:
  call void @output_impl(ptr @output_var)
  ret void
}

define void @obc_check_error() {
entry:
  call void @obc_check_error_impl(ptr @arg0, ptr @arg1, ptr @arg2)
  ret void
}

define void @main() {
entry:
  %x = alloca [8 x i32], align 4
  store [8 x i32] zeroinitializer, ptr %x, align 4
  %index = alloca i32, align 4
  %addr = alloca i32, align 4
  call void @input()
  %0 = load i32, ptr @input_var, align 4
  store i32 %0, ptr %index, align 4
  br label %while.cond

while.cond:                                       ; preds = %while.body, %entry
  %1 = load i32, ptr %index, align 4
  %2 = icmp sge i32 %1, 0
  br i1 %2, label %while.body, label %while.end

while.body:                                       ; preds = %while.cond
  call void @input()
  %3 = load i32, ptr @input_var, align 4
  store i32 %3, ptr %addr, align 4
  %4 = load i32, ptr %addr, align 4
  %5 = load i32, ptr %index, align 4
  %6 = getelementptr inbounds [8 x i32], ptr %x, i32 0, i32 %5
  store i32 %4, ptr %6, align 4
  call void @input()
  %7 = load i32, ptr @input_var, align 4
  store i32 %7, ptr %index, align 4
  br label %while.cond

while.end:                                        ; preds = %while.cond
  ret void
}
\end{minted}

根据属于,我们可以分析出:\verb|x|是一个长度为8的整型数组,但输入的索引溢出了数组边界。第一次输入索引为12,第二次输出索引为13,由于\verb|x|不是obc数组,因此没有进行obc检测。

根据\verb|main|函数的栈帧进行分析,结合x86-linux的函数调用约定,可以推测出输入的两个索引12和13分别覆盖了\verb|main|函数栈帧中的返回地址和旧的基址指针。返回地址被覆盖为\verb|0x4000000|,旧的基址指针被覆盖为\verb|0x0|。
而\verb|0x4000000|很可能不是一个可执行的合法地址,因此当函数返回时,程序试图跳转到这个地址执行代码,导致了\verb|You are hacked|错误。
\section{实验中遇到的问题与挑战}

\begin{itemize}
    \item 在实现obc检测时,发现\verb|runtime.cpp|中的函数签名与obc检测函数中给出的示例不符:前者只有两个参数,而示例中有三个参数。经过查阅资料及询问大语言模型,应该使用三个参数的版本,因此在\verb|runtime.cpp|中添加了第三个参数。
    \item 在实现if和while节点时,发现如果在条件判断块或循环体块中有其他基本块,原来判断终止指令的方法不再适用,因此改为获取当前基本块并判断其是否有终止指令。
\end{itemize}

\section{SafeCIRBuilder.cpp以外的改动}

在\verb|runtime.cpp|中,修改了\verb|obc_check_error|函数的参数列表,添加了第三个参数用于传递数组名称:

\begin{minted}{cpp}
auto obc_check_error_impl = Function::Create(
    FunctionType::get(
        Type::getVoidTy(module->getContext()),
        {Type::getInt32PtrTy(module->getContext()),
        Type::getInt32PtrTy(module->getContext()),
        Type::getInt8PtrTy(module->getContext())},
        false),
    GlobalValue::LinkageTypes::ExternalLinkage,
    "obc_check_error_impl",
    module);
\end{minted}



\end{document}